\documentclass[10pt,letterpaper]{article}

\usepackage{color}
\usepackage{url}
\usepackage{hyperref}
\usepackage{enumitem}
\usepackage{geometry}
\geometry{left=0.75in, right=0.75in, top=0.75in, bottom=0.75in}

\def\name{Xiaoli Sun, Jaydeep Hemant Rotithor}


\begin{document}
\begin{titlepage}
\begin{center}
    \Huge
    \textbf{Project 1}
    
    \vspace{0.5in}
    \large
    CS444 - Operating Systems II\\
    
    \vspace{0.2in}
    \large
    Fall 2017\\
    
    \vspace{0.2in}
    \large
    Instructor: D. Kevin McGrath\\
    
    \vspace{0.2in}
    \textbf{Xiaoli Sun, Jaydeep Hemant Rotithor}
    
    \vspace{0.5in}
    \textbf{Abstract}\\
    \vspace{0.2in}
    This is the first CS444 assignment. There are two parts to this assignment. The first part is to build a kernel and then let the kernel boot on the virtual machine. We do this because we need to learn how to build a kernel and set up a virtual machine. The second part is to do a concurrency exercise using C programming. In this paper, we will document how we build a kernel on the VM in each step, and describe how to design a concurrency program in C. 
    
    \vspace{0.3in}
    \vfill
    %\Large 
    
    Oct 3, 2017

\end{center}
\end{titlepage}




\newpage
\section{Build kernel and run VM}

\textbf{}

\textbf{Step 1:}
mkdir 35\\ \\
The first step is to create a folder under /scratch/fall2017.\\


\textbf{Step 2:}
git clone git://git.yoctoproject.org/linux-yocto-3.19 \\ \\
The second step is to clone the kernel from github to the folder created in step 1.\\

\textbf{Step 3:}
git checkout tags/v3.19.2\\ \\
The third step is to switch tags to v3.19.2 under /scratch/fall2017/35/linux-yocto-3.19 \\

\textbf{Step 4:}
cp config-3.19.2-yocto-standard ./linux-yocto-3.19/.config\\ \\
The fourth step is cd /scratch/files, then copy config-3.19.2-yocto-standard to the directory of the root of linux tree: ./linux-yocto-3.19/.config. After that,  copying bzImage-qemux86.bin and core-image-lsb-sdk-qumex86.ext4 to /scratch/fall2017/35/linux-yocto-3.19 \\

\textbf{Step 5:}
source /scratch/files/environment-setup-i586-poky-linux.csh\\ \\
The fifth step is to cd to linux-yocto-3.19 directory, and source the csh file. \\

\textbf{Step 6:}
make -j4 all\\ \\
We can build the kernel in the sixth step by inputting make -j4 all command. 4 threads takes about 5 minutes to finish. \\

\textbf{Step 7:}
qemu-system-i386 -gdb tcp::5535 -S -nographic -kernel bzImage-qemux86.bin -drive file=core-image-lsb-sdk-qemux86.ext4,if=virtio -enable-kvm -net none -usb -localtime --no-reboot --append "root=/dev/vda rw console=ttyS0 debug"\\ \\
In the seventh step, we changed ???? to 5535 and then we are ready to boot it on the VM. \\

\textbf{Step 8:}
gdb\\ \\
The eighth step is to connect gbd to a remote target at the specified port(5535). First we open a new PUTTY window, connect to the os2 server and change the directory to /scratch/fall2017/35/linux-yocto-3.19. After that, simply entering gdb.\\

\textbf{Step 9:}
target remote :5535\\ \\
The ninth step is to connect gbd to a remote target at the specified port(5535).  After that, entering "file bzImage-qemux86.bin".\\

\textbf{Step 10:}
continue\\ \\
The tenth step is to continue the process and will cause the OS to boot. \\

\textbf{Step 11:}
root\\ \\
In the eleventh step, we change to the first PUTTY window and find that the kernel is already booted on the VM. Then we login in as root and no password is required.\\

\textbf{Step 12:}
shutdown -h now\\ \\
In the twelveth step, we can use the command above to shut down VM.\\
 

\section*{Concurrency solution}

\textbf{}
In this problem, we should first understand that there are two processes, consumer and producer. There's also a buffer shared by the two processes. The producer will produce a number and put it into the buffer. Meanwhile, the consumer will delete this number from buffer. When the buffer is full, block the producer until consumer deletes a number from buffer. When the buffer is empty, block the consumer until producer adds an number to buffer. The problems that we met when solving this question are when and where the mutex should be blocked, how to correctly block consumer and producer, etc. 
 
 
\section{Flags explanation}

\textbf{}

\textbf{-gdb tcp::5535}\\
It tells qemu to wait for gdb connection from port 5535.\\

\textbf{-s}\\ 
Don't start CPU when boot.\\
 
\textbf{-nographic}\\ 
Disable GUIs and use command line. \\
 
\textbf{-kernel}\\
Use bzimage as a boot img.\\
 
\textbf{-drive file=core-image-lsb-sdk-qemux86.ext4}\\
Let qemu use core-image-lsb-sdk-qemux86.ext4 as disk image.\\
 
\textbf{-enable-kvm}\\
Enable the KVM for qemu.\\
 
\textbf{-net none}\\
kernel won't use any network devices.\\

\textbf{-usb}\\
enable usb devices.\\

\textbf{-localtime}\\
set the time as local time.\\

\textbf{-no reboot}\\
no roob when exit kernel\\

\textbf{--append "root=/dev/vda rw console=ttyS0 debug"}\\
use "root=/dev/vda rw console=ttyS0 debug" to run the command line in the linux kernel.\\


\section{Concurrency questions}

\textbf{}

\textbf{1. What do you think the main point of this assignment is?}\\

In my opinion, the main purpose of this assignment is to consolidate my understand of concurrency. Concurrency is a very important concept in operating systems. Modern operating systems allow multiple processes run at the same and access same resources at the same time as well. This assignment also let me practice how to deal with the situation like resources are empty or full when run multiple process access the same resources.\\

\textbf{2. How did you personally approach the problem? Design decisions, algorithm, etc.}\\

First of all, I carefully read the question and found that one structure was required and the structure should contains two integers, a number and a period waiting time. I also created three threads by using pthread, two for consumer and producer, the rest for mutex. Two processes will have a shared resources which could hold 32 items. I created a int array and define the array size as 32. Afterwards, I created two functions: produce and consume. In produce function, I first let producer sleep for a random number of seconds, the time is generated by genrand\_int32() function. The genrand()\_int32() function is defined at "mt19937ar.c" file and I included it in my C code. Then the function will check if the buffer size if full or not. During checking the buffer, the buffer is locked  by using mutex. If the buffer is full, the function will wait for 2 to 9 seconds to add a new number to the buffer until consumer send the signal to producer to inform the buffer is no longer full . Then assign the random generated new number and waiting period to structure. Finally, print the result. The consumer has the same approach as the producer.\\

\textbf{3. How did you ensure your solution was correct? Testing details, for instance.}\\

I tested my code multiple time by changing buffer size. When buffer size is equal to 0, consumer will not delete number but wait for seconds until producer send a signal to it. When buffer size is equal to 32, producer will stop generating new number until it receives signal from consumer. I also testing waiting period to see if consumer and producer actually sleep for the time defined in the code.\\

\textbf{4. What did you learn?}\\

In this assignment, I learned how to correctly run multiple processes simultaneously and access the same resource at the same time. I also learned how to write make file for pdf and how to use Latex. Finally, I learned that CS444 is not all based on writing codes, it's more focus on important concepts of operating systems.


\section{Version Control}

\textbf{}

\begin{tabular}{|l|l|l|p{6cm}|} \hline
Revision & Date & Author(s) & Description\\ \hline
7909a21 & 10.5.17 & Xiaoli Sun & Created central repository.\\ \hline
3ad1310 & 10.5.17 & Xiaoli Sun & Upload concurrency C code.\\ \hline
8052f07 & 10.5.17 & Xiaoli Sun & Upload all files for concurrency.\\ \hline
5851fc3 & 10.5.17 & Xiaoli Sun & Add pdf file and source file for write up.\\ \hline

\end{tabular}



\section{Work log}

\textbf{}

\begin{tabular}{|l|p{12cm}|} \hline
Date & Summary\\ \hline
Friday, 9/29 & Looked over project requirements and began planning.\\ \hline
Saturday, 9/30 &\\ \hline
Sunday, 10/1 &\\ \hline
Monday, 10/2 & Created the group repository under scratch/fall2017 and cloned the linux-yocto-3.19 repository. \\\hline
Tuesday, 10/3 &  Successfully ran qemu in debug mode with GDB. \\ \hline
Wednesday, 10/4 & Began write-up. \\ \hline
Thursday, 10/5 & Updated the .config file, compiled the kernel, and successfully got it to run. \\ \hline
Friday, 10/6 & Finished write-up and prepare to submit the assignment.\\ \hline
Saturday, 10/7 & Fixing LaTeX Error when compiling latex source code. \\ \hline

\end{tabular}

\end{document}